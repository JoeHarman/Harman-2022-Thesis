
The regulation of gene expression is a complex process that confers cell identity and drives cell differentiation. Transcription factors (TFs) are an important modulator of gene expression and are critical to understanding the regulatory code characterising cell behaviour. However, TFs function in cooperation with one another, acting as a complex combinatorial code to enact gene regulation. Gene regulatory network (GRN) models are used to describe the regulation of genes and to account for this combinatorial code. When the regulatory code underpinning normal cell behaviour goes awry this can lead to abnormal GRNs and the development of disease. This poses two questions: how does TF cooperation drive cellular processes, and how are normal GRNs deregulated in disease?

Haematopoietic stem cells are first established in the developing embryo through a process known as the endothelial-to-haematopoietic transition (EHT). EHT involves dynamic gene regulation across well-defined populations, and is critically dependent on the master TF regulator, RUNX1. EHT is therefore a fascinating system to study TF regulation in driving a complex developmental process. When haematopoietic GRNs are misregulated this can lead to disease. A common childhood leukaemia involves chromosomal rearrangements resulting in novel Mixed Lineage Leukaemia (MLL) fusion proteins that aberrantly drive transcription. MLL-rearranged leukaemias are considered a wholly transcriptional disease, and as such are uniquely suited for studying through GRN approaches.

In this thesis, the study of GRNs underpinning normal and malignant haematopoietic processes revealed insights into the role of TF cooperation. Notably, Runx1 cooperates with TFs to drive repression of a transiently activated network during EHT. In MLL-AF4 leukaemia, MLL-AF4 drives TF cooperation by altering the chromatin environment and encouraging local TF binding, and within this MLL-AF4 context RUNX1 activity is biased towards activation. RUNX1 was also found to cooperate with MYB to synergistically activate \textit{BCL2} and \textit{MYC}, promoting leukaemogenesis.